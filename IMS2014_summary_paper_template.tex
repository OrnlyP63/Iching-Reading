%% paper_template.tex is a modification of:
%% bare_conf.tex 
%% V1.2
%% 2002/11/18
%% by Michael Shell
%% mshell@ece.gatech.edu
%% 
%% This is a skeleton file demonstrating the use of IEEEtran.cls 
%% (requires IEEEtran.cls version 1.6b or later) with an IEEE conference paper.
%% 
%% Support sites:
%% http://www.ieee.org
%% and/or
%% http://www.ctan.org/tex-archive/macros/latex/contrib/supported/IEEEtran/ 
%%
%% This code is offered as-is - no warranty - user assumes all risk.
%% Free to use, distribute and modify.

% *** Authors should verify (and, if needed, correct) their LaTeX system  ***
% *** with the testflow diagnostic prior to trusting their LaTeX platform ***
% *** with production work. IEEE's font choices can trigger bugs that do  ***
% *** not appear when using other class files.                            ***
% Testflow can be obtained at:
% http://www.ctan.org/tex-archive/macros/latex/contrib/supported/IEEEtran/testflow


% Note that the a4paper option is mainly intended so that authors in
% countries using A4 can easily print to A4 and see how their papers will
% look in print. Authors are encouraged to use U.S. letter paper when 
% submitting to IEEE. Use the testflow package mentioned above to verify
% correct handling of both paper sizes by the author's LaTeX system.
%
% Also note that the "draftcls" or "draftclsnofoot", not "draft", option
% should be used if it is desired that the figures are to be displayed in
% draft mode.
%
% This paper can be formatted using the % (instead of conference) mode.
%++++++++++++++++++++++++++++++++++++++++++++++++++++++
%\documentclass[conference]{IEEEims} % Modified for MTT-IMS
%\documentclass[conference]{IMSTemplate}
\documentclass[conference]{IEEEtran}
%++++++++++++++++++++++++++++++++++++++++++++++++++++++
% If the IEEEtran.cls has not been installed into the LaTeX system files, 
% manually specify the path to it:
% \documentclass[conference]{../sty/IEEEtran} 


% some very useful LaTeX packages include:

%\usepackage{cite}      % Written by Donald Arseneau
                        % V1.6 and later of IEEEtran pre-defines the format
                        % of the cite.sty package \cite{} output to follow
                        % that of IEEE. Loading the cite package will
                        % result in citation numbers being automatically
                        % sorted and properly "ranged". i.e.,
                        % [1], [9], [2], [7], [5], [6]
                        % (without using cite.sty)
                        % will become:
                        % [1], [2], [5]--[7], [9] (using cite.sty)
                        % cite.sty's \cite will automatically add leading
                        % space, if needed. Use cite.sty's noadjust option
                        % (cite.sty V3.8 and later) if you want to turn this
                        % off. cite.sty is already installed on most LaTeX
                        % systems. The latest version can be obtained at:
                        % http://www.ctan.org/tex-archive/macros/latex/contrib/supported/cite/

%\usepackage{graphicx}  % Written by David Carlisle and Sebastian Rahtz
                        % Required if you want graphics, photos, etc.
                        % graphicx.sty is already installed on most LaTeX
                        % systems. The latest version and documentation can
                        % be obtained at:
                        % http://www.ctan.org/tex-archive/macros/latex/required/graphics/
                        % Another good source of documentation is "Using
                        % Imported Graphics in LaTeX2e" by Keith Reckdahl
                        % which can be found as esplatex.ps and epslatex.pdf
                        % at: http://www.ctan.org/tex-archive/info/
% NOTE: for dual use with latex and pdflatex, instead load graphicx like:
%\ifx\pdfoutput\undefined
%\usepackage{graphicx}
%\else
%\usepackage[pdftex]{graphicx}
%\fi
%+++++++++++++++++++++++++++++++++++++++++++
% Added to commands
\input epsf
\usepackage{graphicx}
%\usepackage[UTF8]{ctex}
%+++++++++++++++++++++++++++++++++++++++++++
% However, be warned that pdflatex will require graphics to be in PDF
% (not EPS) format and will preclude the use of PostScript based LaTeX
% packages such as psfrag.sty and pstricks.sty. IEEE conferences typically
% allow PDF graphics (and hence pdfLaTeX). However, IEEE journals do not
% (yet) allow image formats other than EPS or TIFF. Therefore, authors of
% journal papers should use traditional LaTeX with EPS graphics.
%
% The path(s) to the graphics files can also be declared: e.g.,
% \graphicspath{{../eps/}{../ps/}}
% if the graphics files are not located in the same directory as the
% .tex file. This can be done in each branch of the conditional above
% (after graphicx is loaded) to handle the EPS and PDF cases separately.
% In this way, full path information will not have to be specified in
% each \includegraphics command.
%
% Note that, when switching from latex to pdflatex and vice-versa, the new
% compiler will have to be run twice to clear some warnings.


%\usepackage{psfrag}    % Written by Craig Barratt, Michael C. Grant,
                        % and David Carlisle
                        % This package allows you to substitute LaTeX
                        % commands for text in imported EPS graphic files.
                        % In this way, LaTeX symbols can be placed into
                        % graphics that have been generated by other
                        % applications. You must use latex->dvips->ps2pdf
                        % workflow (not direct pdf output from pdflatex) if
                        % you wish to use this capability because it works
                        % via some PostScript tricks. Alternatively, the
                        % graphics could be processed as separate files via
                        % psfrag and dvips, then converted to PDF for
                        % inclusion in the main file which uses pdflatex.
                        % Docs are in "The PSfrag System" by Michael C. Grant
                        % and David Carlisle. There is also some information 
                        % about using psfrag in "Using Imported Graphics in
                        % LaTeX2e" by Keith Reckdahl which documents the
                        % graphicx package (see above). The psfrag package
                        % and documentation can be obtained at:
                        % http://www.ctan.org/tex-archive/macros/latex/contrib/supported/psfrag/

%\usepackage{subfigure} % Written by Steven Douglas Cochran
                        % This package makes it easy to put subfigures
                        % in your figures. i.e., "figure 1a and 1b"
                        % Docs are in "Using Imported Graphics in LaTeX2e"
                        % by Keith Reckdahl which also documents the graphicx
                        % package (see above). subfigure.sty is already
                        % installed on most LaTeX systems. The latest version
                        % and documentation can be obtained at:
                        % http://www.ctan.org/tex-archive/macros/latex/contrib/supported/subfigure/

%\usepackage{url}       % Written by Donald Arseneau
                        % Provides better support for handling and breaking
                        % URLs. url.sty is already installed on most LaTeX
                        % systems. The latest version can be obtained at:
                        % http://www.ctan.org/tex-archive/macros/latex/contrib/other/misc/
                        % Read the url.sty source comments for usage information.

%\usepackage{stfloats}  % Written by Sigitas Tolusis
                        % Gives LaTeX2e the ability to do double column
                        % floats at the bottom of the page as well as the top.
                        % (e.g., "\begin{figure*}[!b]" is not normally
                        % possible in LaTeX2e). This is an invasive package
                        % which rewrites many portions of the LaTeX2e output
                        % routines. It may not work with other packages that
                        % modify the LaTeX2e output routine and/or with other
                        % versions of LaTeX. The latest version and
                        % documentation can be obtained at:
                        % http://www.ctan.org/tex-archive/macros/latex/contrib/supported/sttools/
                        % Documentation is contained in the stfloats.sty
                        % comments as well as in the presfull.pdf file.
                        % Do not use the stfloats baselinefloat ability as
                        % IEEE does not allow \baselineskip to stretch.
                        % Authors submitting work to the IEEE should note
                        % that IEEE rarely uses double column equations and
                        % that authors should try to avoid such use.
                        % Do not be tempted to use the cuted.sty or
                        % midfloat.sty package (by the same author) as IEEE
                        % does not format its papers in such ways.

%\usepackage{amsmath}   % From the American Mathematical Society
                        % A popular package that provides many helpful commands
                        % for dealing with mathematics. Note that the AMSmath
                        % package sets \interdisplaylinepenalty to 10000 thus
                        % preventing page breaks from occurring within multiline
                        % equations. Use:
%\interdisplaylinepenalty=2500
                        % after loading amsmath to restore such page breaks
                        % as IEEEtran.cls normally does. amsmath.sty is already
                        % installed on most LaTeX systems. The latest version
                        % and documentation can be obtained at:
                        % http://www.ctan.org/tex-archive/macros/latex/required/amslatex/math/



% Other popular packages for formatting tables and equations include:

%\usepackage{array}
% Frank Mittelbach's and David Carlisle's array.sty which improves the
% LaTeX2e array and tabular environments to provide better appearances and
% additional user controls. array.sty is already installed on most systems.
% The latest version and documentation can be obtained at:
% http://www.ctan.org/tex-archive/macros/latex/required/tools/

% Mark Wooding's extremely powerful MDW tools, especially mdwmath.sty and
% mdwtab.sty which are used to format equations and tables, respectively.
% The MDWtools set is already installed on most LaTeX systems. The lastest
% version and documentation is available at:
% http://www.ctan.org/tex-archive/macros/latex/contrib/supported/mdwtools/


% V1.6 of IEEEtran contains the IEEEeqnarray family of commands that can
% be used to generate multiline equations as well as matrices, tables, etc.


% Also of notable interest:

% Scott Pakin's eqparbox package for creating (automatically sized) equal
% width boxes. Available:
% http://www.ctan.org/tex-archive/macros/latex/contrib/supported/eqparbox/



% Notes on hyperref:
% IEEEtran.cls attempts to be compliant with the hyperref package, written
% by Heiko Oberdiek and Sebastian Rahtz, which provides hyperlinks within
% a document as well as an index for PDF files (produced via pdflatex).
% However, it is a tad difficult to properly interface LaTeX classes and
% packages with this (necessarily) complex and invasive package. It is
% recommended that hyperref not be used for work that is to be submitted
% to the IEEE. Users who wish to use hyperref *must* ensure that their
% hyperref version is 6.72u or later *and* IEEEtran.cls is version 1.6b 
% or later. The latest version of hyperref can be obtained at:
%
% http://www.ctan.org/tex-archive/macros/latex/contrib/supported/hyperref/
%
% Also, be aware that cite.sty (as of version 3.9, 11/2001) and hyperref.sty
% (as of version 6.72t, 2002/07/25) do not work optimally together.
% To mediate the differences between these two packages, IEEEtran.cls, as
% of v1.6b, predefines a command that fools hyperref into thinking that
% the natbib package is being used - causing it not to modify the existing
% citation commands, and allowing cite.sty to operate as normal. However,
% as a result, citation numbers will not be hyperlinked. Another side effect
% of this approach is that the natbib.sty package will not properly load
% under IEEEtran.cls. However, current versions of natbib are not capable
% of compressing and sorting citation numbers in IEEE's style - so this
% should not be an issue. If, for some strange reason, the user wants to
% load natbib.sty under IEEEtran.cls, the following code must be placed
% before natbib.sty can be loaded:
%
% \makeatletter
% \let\NAT@parse\undefined
% \makeatother
%
% Hyperref should be loaded differently depending on whether pdflatex
% or traditional latex is being used:
%
%\ifx\pdfoutput\undefined
%\usepackage[hypertex]{hyperref}
%\else
%\usepackage[pdftex,hypertexnames=false]{hyperref}
%\fi
%
% Pdflatex produces superior hyperref results and is the recommended
% compiler for such use.



% *** Do not adjust lengths that control margins, column widths, etc. ***
% *** Do not use packages that alter fonts (such as pslatex).         ***
% There should be no need to do such things with IEEEtran.cls V1.6 and later.


% correct bad hyphenation here
\hyphenation{op-tical net-works semi-conduc-tor IEEEtran}
\begin{document}

% paper title
%\title{Submission Format for IMS2014 (Title in 24-point Times font)}
% If the \LARGE is deleted, the title font defaults to  24-point.
% Actually, 
% the \LARGE sets the title at 17 pt, which is close enough to 18-point.
%+++++++++++++++++++++++++++++++++++++++++++
\title{\LARGE Extract news topic  of "I Ching" divination with Topic analysis}
%+++++++++++++++++++++++++++++++++++++++++++
% author names and affiliations
% use a multiple column layout for up to three different
% affiliations
%+++++++++++++++++++++++++++++++++++++++++++
%\author{\authorblockN{J. Clerk Maxwell}
%\authorblockA{School of Electrical and\\Computer Engineering\\
%Somewhere Institute of Technology\\
%City, State 54321--0000\\
%Email: maxwell@curl.edu}
%\and
%\authorblockN{Michael Faraday}
%\authorblockA{(List authors on this line using 12 point Times font\\ - use a second line if necessary)\\
%Microwave Research\\
%City, State/Region, Mail/Zip Code, Country\\
%Email: homer@thesimpsons.com}
%\and
%\authorblockN{Andr\'e M. Amp\`ere \\ }
%\authorblockA{Starfleet Academy\\
%San Francisco, CA 96678-2391\\
%Telephone: (800) 555--1212\\
%Fax: (888) 555--1212}}

\author{\authorblockN{Phiphat Chomchit 630631028}
\authorblockA{\authorrefmark{1}Chiang Mai University, Data science, Department of Engineer.}}


%+++++++++++++++++++++++++++++++++++++++++++++++++++

% avoiding spaces at the end of the author lines is not a problem with
% conference papers because we don't use \thanks or \IEEEmembership


% for over three affiliations, or if they all won't fit within the width
% of the page, use this alternative format:
% 
% Another example.
%\author{\authorblockN{Michael Shell\authorrefmark{1},
%Homer Simpson\authorrefmark{2},
%James Kirk\authorrefmark{3}, 
%Montgomery Scott\authorrefmark{3} and
%Eldon Tyrell\authorrefmark{4}}
%\authorblockA{\authorrefmark{1}School of Electrical and Computer Engineering\\
%Georgia Institute of Technology,
%Atlanta, Georgia 30332--0250\\ Email: mshell@ece.gatech.edu}
%\authorblockA{\authorrefmark{2}Twentieth Century Fox, Springfield, USA\\
%Email: homer@thesimpsons.com}
%\authorblockA{\authorrefmark{3}Starfleet Academy, San Francisco, California 96678-2391\\
%Telephone: (800) 555--1212, Fax: (888) 555--1212}
%\authorblockA{\authorrefmark{4}Tyrell Inc., 123 Replicant Street, Los Angeles, California 90210--4321}}



% use only for invited papers
%\specialpapernotice{(Invited Paper)}

% make the title area
\maketitle

\begin{abstract}
...
\end{abstract}
\IEEEoverridecommandlockouts
\begin{keywords}
Divination, I Ching, Topic analysis, self-organizing map.
\end{keywords}
% no keywords

% For peer review papers, you can put extra information on the cover
% page as needed:
% \begin{center} \bfseries EDICS Category: 3-BBND \end{center}
%
% for peerreview papers, inserts a page break and creates the second title.
% Will be ignored for other modes.
\IEEEpeerreviewmaketitle



\section{Introduction}
% no \PARstart
China’s rapid economic development in the 21st Century created a global impact. As a result, Chinese culture started attracting increasing attention from Western countries \cite{1}. China's rise to power has affected business, politics, and academia, making the culture and knowledge of China is getting more and more attention. Understanding Chinese culture is therefore becoming important. It is interesting to study the behavior of people influenced by Chinese culture. One of the fascinating Chinese cultures is divination called “I Ching”.

I Ching is one of the bases of major ancient Chinese knowledge.
\cite{2}described the I Ching as ‘a must-read for any ruler’ in The Book of Han. In China, the Four Books and Five Classics (Sishu Wujing) are considered to occupy noble, historical, and cultural positions, whilst the I Ching, seen as ‘the first among the classics’, was greatly respected by ancient Chinese statesmen and the literati.

\cite{3}explained in ‘the Commentaries of the I Ching’ that ‘I Ching is used as a mention to make decisions in the world, handle human relationships, and clarify methods for governing. They declared that the principles of deduction and inference establish in the I Ching could be widely used in governance, politics, management, and human relations as main directions for future actions.
In conclusion, I Ching studied the environment to separate things. And find the balance of that system.

Although the Chinese regard the theories contained in the I Ching as proven intellectual principles, the Western-educated generations feel that there has been a lack of empirical research on the I Ching using contemporary and scientific methods \cite{5}.

However, the rise of China's power has made education in I Ching more interesting. In academic fields has developed I Ching for use in their respective fields.
The I Ching, a classic part of Chinese culture, gradually drew the attention of scholars both at home and abroad \cite{1}. 
Moreover, Theories from the I Ching have drawn cumulative attention internationally in recent years and relevant studies and applications of these have emerged in various disciplines. \cite{4}


\cite{1} compared the differences between the decision-making models of the I Ching and Western management theories, as well as rational, bounded-rational, intuitive, implicit-favorite, and garbage-can decision-making models, and claimed that the decision-making model of the I Ching is appropriate to management practices and provides a reference for decision making when complete information is unavailable.

Because the I Ching knowledge has been used and developed from the more powerful countries of the world (the world's largest buyer) and has long been interested in I Ching. So education to understand the development of I Ching knowledge to the world is important.


%\begin{table*}
%\centerline { TABLE 1  } 
%\vskip5pt
%\centerline { Summary of Typographical Settings}
%\vskip2pt
%\centerline{
%\vbox{\offinterlineskip
%\hrule
%%\vskip2pt\hrule\vskip2pt
%% Leading & means preamble template repeats infinitely. p.241 TeX Book.
%\halign{&\vrule#&
%\strut\quad#\hfil\quad\cr
%%Use either first and third lines following this description, OR the
%%second line.  The first choice is used when all vertical rules go to the
%%top of the first horizontal line of the table.  The second choice below
%%(with the \strut) is used when there are column headings that span
%%more than one column.  The \strut in that column line will not have the
%%vertical tic marks in the horizontal rule.  Note that a vrule is also
%%considered a column, so when using \multispanx, x is the number of
%%all columns including the ``vrule.'' 
%%height2pt&\omit&&\omit&&\omit&&\omit&&\omit&&\omit&&\omit&&\omit&&\omit&\cr
%&\strut &&\multispan5\hfil {\bf Font Specifics}\hfil&&\multispan9\hfil {\bf Paragraph Description}\hfil &\cr
%%&\omit &&\multispan5\hfil {\bf Font Specifics}\hfil&&\multispan9\hfil {\bf Paragraph Description}\hfil &\cr
%&{\bf Section}&&\multispan5\hfil (Times Roman unless
%specified)\hfil&&\multispan5\hfil spacing (in points)\hfil &&
%alignment&&indent&\cr
%&\omit&&style&&size&&special&&line&&before&&after&&\omit&&(in inches)&\cr
%height2pt&\omit&&\omit&&\omit&&\omit&&\omit&&\omit&&\omit&&\omit&&\omit&\cr
%\noalign{\hrule}
%height2pt&\omit&&\omit&&\omit&&\omit&&\omit&&\omit&&\omit&&\omit&&\omit&\cr
%%\noalign{\vskip2pt\hrule\vskip2pt}
%%\omit&\omit&\omit&\omit\cr
%&Title&&plain&&18&&none&&single&&6&&6&&centered&&none&\cr
%&Autohr List&&plain&&12&&mpme&&single&&6&&6&&centered&&none&\cr
%&Affiliations&&plain&&12&&none&&single&&6&&6&&centered&&none&\cr
%&Abstract&&bold&&9&&none&&exactly 10&&0&&0&&justified&&0.125 $1^{st}$ line&\cr
%&Headings&&plain&&10&&small caps&&exactly 12&&18&&6&&centered&&none&\cr
%&Subheadings&&italic&&10&&none&&exactly 12&&6&&6&&left&&none&\cr
%&Body&&plain&&10&&none&&exactly 12&&0&&0&&justified&&0.125 $1^{st}$ line&\cr
%&Paragrahps&&\omit&&\omit&&\omit&&\omit&&\omit&&\omit&&\omit&&\omit&\cr
%&Equations&&\multispan5 \hfil Symbol font for special characters
%\hfil&&single&&6&&6&&centered&&none&\cr
%&Figures&&\multispan5 \hfil 6 to 9 point sans serif (Helvetica)\hfil&&single&&0&&0&&centered&&none&\cr
%&Figure Captions&&plain&&9&&none &&10&&0&&0&&justified&&none, tab at 0.5&\cr
%&References&&plain&&9&&none&&10&&0&&0&&justified&&0.25 hanging&\cr
%height2pt&\omit&&\omit&&\omit&&\omit&&\omit&&\omit&&\omit&&\omit&&\omit&\cr}
%\hrule}}
%\end{table*}

\section{Literature review}
\subsection{History and Structure}
The I Ching or Yi Jing is an ancient Chinese divination text and among the ancient of the Chinese classics. In the first place a divination manual in the Western Zhou period (1000–750 BC), over the direction of the Warring States period and early imperial period (500–200 BC) it was converted into a cosmological text with a series of philosophical commentaries known as the "Ten Wings".\cite{20}
The essence of the I Ching is a Western Zhou divination text called the Changes of Zhou .\cite{21} Various contemporary scholars indicate dates ranging between the 10th and 4th centuries BC for the assembly of the text in approximately its current form.\cite{22} 
Hexagram
In the canonical I Ching, the hexagrams are arranged in an order dubbed the King Wen sequence after King Wen of Zhou, who founded the Zhou dynasty and supposedly reformed the method of interpretation. The sequence generally pairs hexagrams with their upside-down equivalents, although in eight cases hexagrams are paired with their inversion.\cite{23} 
Yin and yang are represented by broken and solid lines: yin is broken (- -) and yang is solid (---). Different constructions of three yin and yang lines lead to eight trigrams namely, Qian , Dui , Li , Zhen , Xun , Kan , Gen , Kun.
The different combinations of the two trigrams lead to 64 hexagrams, it is called Bagua.

\cite{5} pointed out that the I Ching naming system is easy to apply because it requires only the stroke-count of each character and therefore has great potential for future development in talent selection processes. With various combinations of Bagua, Yinyang, and Wu Xing, the I Ching naming system, made up of 64 hexagrams (8 x 8), can be employed as a reference to classify individuals’ traits and social status as well as determine their future development. Using permutations and combinations of sheng and ke of Wu
Xing as its technique base, the I Ching naming system generates a classification of people’s traits and social status by interpreting relationships among human beings, physical objects,
and the environment by applying the concepts of tiange (heavenly traits), renge (personal
traits), dige (earth traits), waige (external traits), and zongge (overall traits).

the theories of the I Ching and Bagua can evolve with time and space, take any form in the world, and be converted into any fundamental concept related to human beings, temporal events, or physical objects \cite{4}

Mutual creation refers to the fact that one element provides support for the development of a second element, that is, among the five elements, metal enriches water, water nourishes wood, wood feeds fire, fire creates
earth (ash), earth bears metal, metal further enriches water, and hence, an endless, vitalized cycle. Mutual prevalence refers to the fact that one element represses another to maintain balance among the five elements; specifically, metal chops wood, wood parts earth, earth sink water, water extinguishes fire, and fire melts metal, and thus, a second endless cycle continues \cite{5}.

\subsection{I Ching in academic areas.}
For example, how to improve I Ching in academic areas: 

*I-Ching Divination Evolutionary Algorithm and its Convergence Analysis; They used the I Ching system to solve optimization problem.

*Design of Highly Nonlinear Substitution Boxes Based on I-Ching Operators; They design substitution boxes (S-Boxes) using innovative I-Ching operators (ICOs) that have evolved from ancient Chinese I-Ching philosophy.

*Thinking and Modeling for Big Data from the Perspective of the I Ching; They propose the big
data thinking and modeling techniques from the perspective of the I Ching theory.

\subsection{Fortune telling online service.}
\cite{6}reported that 52\% of the 1,079 surveyed respondents in her survey indicated that they consulted fortune tellers. Another scholar reported that 45% of
surveyed couples said that they consulted fortune tellers prior to marriage in order to
find out whether or not their horoscopes were a good match \cite{7}. Based on a
longitudinal study of three national sample surveys, \cite{8}pointed out that
uncertainties in domestic political and economic situations were a significant
contributing factor in influencing Taiwanese people to seek out fortune telling services.

According to Kasikorn Thai Research Center \cite{10}The
fortune telling market worth 56 Million USD and it is a growing market. Female
were majority who used fortune teller service with 63% compare to male 37%
\cite{11}While a generation Y or millennial was a group of people who
had the most impact for online fortune telling service industry as they used internet the most \cite{9}.

Generation Y and millennials had the most impact on online fortune telling service industry as they used internet the most \cite{9}.


\subsection{Customer Behavior.}
The reviews and ratings are the value-added factors for
other buyers on the Internet as well \cite{13}. The information obtained from
reviews, ratings, and recommendation in online community enhances customers’ knowledge of the product and is the element that handholds customers from the decision process and making actual purchases \cite{14}.

Consumers often make purchase decisions from the information that they
gather. The information search process is costly and time consuming as there are
tradeoffs between perceived cost and benefits of additional search \cite{15}.

If a product or a service is in a wide range of choices, customers know that there are tradeoffs between effort and accuracy\cite{16}.

Experience goods, such as music, are products that require sampling to
evaluate its quality,\cite{17}. For experience goods, it is a matter of personal
taste and falls on a more difficult side to be evaluated effectively. Its nature is
different from search goods, which is an easier product to be evaluated and compared
with objective manners \cite{18}The example of search goods are
automobiles and computers.

\section{Methodology}

The method of this study started from secondary research to investigate an
overview of I Ching divination online news with Topic Analysis. 
Secondary research was used as the preliminary study method to examine the
outline of the fortune-telling communities, for customer behavior, media profile, source of information for reviews, and number of views. The desk research also explored the effect of I Ching culture in online platforms, as well as the behavior of people toward online fortune telling topic.

Data set 
The data set is collected by scraping a web framework that is about the post of news and talking about I Ching from users in ICHING . The data set contain news post, headline, date, number of views, and last active postdate.

Exploratory Data Analysis (EDA)
The data set has 3 type data. There are numerical data, non-numerical data, and unstructured data.
Numerical data\\
- Number of views.\\
Non-numerical data.\\
- Postdate.\\
- Last activation date.\\
Unstructured data\\
- Headline.\\
- Post of news or talking.\\

Preparing Data, The purpose of preparing data is to clean data and transform data to cluster the topic of posts with the LDA  model.

\subsection{Model}
Bag of Word
The Bag-of-words model is a disordered document representation — only the counts of words matter.



\subsubsection{TF-IDF}
NLP (Natural Language Processing) is a subfield of Artificial Intelligence. The textual representation can not be worked with Machine Learning Algorithm and must be converted into some form of numeric representation. The most common approach to deal with that case is Bag of Words (BOW). But BOW cannot perform so well. So, another approach, TF-IDF is used.

Term Frequency $(TF)$ Equation can be described as
$TF(i, j)$ = number of $i$ in the $j$ / total number of words in $j$
Where, $i$ = word, $j$ = document

Inverse Data Frequency $(IDF)$
$IDF (i)$ = $log$(number of $j$ / number of $j$ that contains $i$)
Where, $i$ = word, $j$ = document
The score of TFIDF is by multiplying these two equations.


\subsection{Topic analysis}
Topic analysis (also called topic detection, topic modeling, or topic extraction) is a machine learning technique that organizes and understands large collections of text data, by assigning “tags” or categories according to each individual text’s topic or theme. machine-learning algorithms that count words and find and group similar word patterns.
Topic modeling is an unsupervised machine learning technique. This means it can infer patterns and cluster similar expressions without needing to define topic tags or train data beforehand. This type of algorithm can be applied quickly and easily, but there’s a downside – they are rather inaccurate.

\subsubsection{latent Dirichlet allocation} 
latent Dirichlet allocation (LDA) is a generative statistical model that allows sets of observations to be explained by unobserved groups that explain why some parts of the data are similar.




%\begin{figure}
%\includegraphics{figure1.pdf}
%%\epsfxsize=3.25in\epsfbox{figure1.epsi}
%\caption{ Estimated relationship between the time an author spends reading these instructions and the quality of the author's digest article.}
%\end{figure}


\section{Result and Discussion}
\subsection{Result}
\subsubsection{Using LDA for extract topic from I ching news webpage}
We used LDA to extract topic from 481 subjects. By dividing it into 4 cluster, the key words are as follows
\begin{center}
	\begin{tabular}{ |c|c| } 
		\hline
		Toppic & Important words  \\
		\hline 
		Toppic 0 & china, friend thank, hi, offer, trigram, write\\
		\hline 
		Toppic 1 & journal, app, software, course, pdf, window \\ 
		\hline
		Toppic 2 & amazon, download, html, publish, follow, forum  \\ 
		\hline
		Toppic 3 & order, english, blog, youtube, classic, workshop \\ 
		\hline
	\end{tabular}
\end{center}

Now, we have 4 clusters to define news subjects. Next we will label the topic to all clusters.

\begin{center}
	\begin{tabular}{ |c|c| } 
		\hline
		Toppic & Label  \\
		\hline 
		Toppic 0 & Greeting\\
		\hline 
		Toppic 1 & Online content (book) \\ 
		\hline
		Toppic 2 & Order and download book in Amazon\\ 
		\hline
		Toppic 3 & Online content (video) \\ 
		\hline
	\end{tabular}
\end{center}

The most important cluster is topic 2 and other topic are appear less than 10 persent for each subject.\\

\subsubsection{Using LDA for extract topic from I ching reading borad}
We used LDA to extract topic from 15997 subjects in reading borad. By dividing it into 4 cluster, the key words are as follows
\begin{center}
	\begin{tabular}{ |c|c| } 
		\hline
		Toppic & Important words  \\
		\hline 
		Toppic 0 & relationship, answer, hexagram, interprete, receive\\
		\hline 
		Toppic 1 & hexagram, mean, year, answer ,want \\ 
		\hline
		Toppic 2 & job, work, hexagram, want, good\\ 
		\hline
		Toppic 3 & hexagram, edit, last, move, good\\ 
		\hline
	\end{tabular}
\end{center}

Now, we have 4 clusters to define news subjects. Next we will label the topic to all clusters.

\begin{center}
	\begin{tabular}{ |c|c| } 
		\hline
		Toppic & Label  \\
		\hline 
		Toppic 0 & Relationship\\
		\hline 
		Toppic 1 & Interpret hexagram \\ 
		\hline
		Toppic 2 & Job\\ 
		\hline
		Toppic 3 & How to do when get a prediction \\ 
		\hline
	\end{tabular}
\end{center}
Topic 0 appear in 4349 subjects.\\
Topic 1 appear in 3606 subjects.\\
Topic 2 appear in 4983 subjects.\\
Topic 3 appear in 2850 subjects.\\
%Further information on LaTeX and TeX can be found in \cite{IEEEhowto:kopka} - \cite{knuth}. 
\subsection{Discussion}


% The following statement makes the two columns on the last page more
% or less of equal length.  Placement of this command is by trial and error.
\vfil\eject
%\subsection{Subsection Heading Here}


%\subsubsection{Subsubsection Heading Here}


% Reminder: the "draftcls" or "draftclsnofoot", not "draft", class option
% should be used if it is desired that the figures are to be displayed while
% in draft mode.

% An example of a floating figure using the graphicx package.
% Note that \label must occur AFTER (or within) \caption.
% For figures, \caption should occur after the \includegraphics.
%
%\begin{figure}
%\centering
%\includegraphics[width=2.5in]{myfigure}
% where an .eps filename suffix will be assumed under latex, 
% and a .pdf suffix will be assumed for pdflatex
%\caption{Simulation Results}
%\label{fig_sim}
%\end{figure}


% An example of a double column floating figure using two subfigures.
%(The subfigure.sty package must be loaded for this to work.)
% The subfigure \label commands are set within each subfigure command, the
% \label for the overall fgure must come after \caption.
% \hfil must be used as a separator to get equal spacing
%
%\begin{figure*}
%\centerline{\subfigure[Case I]{\includegraphics[width=2.5in]{subfigcase1}
% where an .eps filename suffix will be assumed under latex, 
% and a .pdf suffix will be assumed for pdflatex
%\label{fig_first_case}}
%\hfil
%\subfigure[Case II]{\includegraphics[width=2.5in]{subfigcase2}
% where an .eps filename suffix will be assumed under latex, 
% and a .pdf suffix will be assumed for pdflatex
%\label{fig_second_case}}}
%\caption{Simulation results}
%\label{fig_sim}
%\end{figure*}



% An example of a floating table. Note that, for IEEE style tables, the 
% \caption command should come BEFORE the table. Table text will default to
% \footnotesize as IEEE normally uses this smaller font for tables.
% The \label must come after \caption as always.
%
%\begin{table}
%% increase table row spacing, adjust to taste
%\renewcommand{\arraystretch}{1.3}
%\caption{An Example of a Table}
%\label{table_example}
%\begin{center}
%% Some packages, such as MDW tools, offer better commands for making tables
%% than the plain LaTeX2e tabular which is used here.
%\begin{tabular}{|c||c|}
%\hline
%One & Two\\
%\hline
%Three & Four\\
%\hline
%\end{tabular}
%\end{center}
%\end{table}
%\begin{table}
%\caption{An Example of a Table}
%\label{table_example}
%\begin{center}
%\begin{tabular}{|c||c|}
%\hline
%One & Two\\
%\hline
%Three & Four\\
%\hline
%\end{tabular}
%\end{center}
%\end{table}

\section{Conclusion}


% conference papers do not normally have an appendix

% use section* for acknowledgment
\section*{Acknowledgment}


% optional entry into table of contents (if used)
%\addcontentsline{toc}{section}{Acknowledgment}


% trigger a \newpage just before the given reference
% number - used to balance the columns on the last page
% adjust value as needed - may need to be readjusted if
% the document is modified later
%\IEEEtriggeratref{8}
% The "triggered" command can be changed if desired:
%\IEEEtriggercmd{\enlargethispage{-5in}}

% references section
% NOTE: BibTeX documentation can be easily obtained at:
% http://www.ctan.org/tex-archive/biblio/bibtex/contrib/doc/

% can use a bibliography generated by BibTeX as a .bbl file
% standard IEEE bibliography style from:
% http://www.ctan.org/tex-archive/macros/latex/contrib/supported/IEEEtran/bibtex
%\bibliographystyle{IEEEtran.bst}
% argument is your BibTeX string definitions and bibliography database(s)
%\bibliography{IEEEabrv,../bib/paper}
%
% <OR> manually copy in the resultant .bbl file
% set second argument of \begin to the number of references
% (used to reserve space for the reference number labels box)
\begin{thebibliography}{1}


%\bibitem {cantrell1}
%W. H. Cantrell, ``Tuning analysis for the high-Q class-E power
%amplifier,'' \emph{IEEE Trans. Microwave Theory \& Tech.}, vol. 48,
%no. 12, pp. 2397-2402, December 2000.
%
%\bibitem {cantrell2}
%W. H. Cantrell, and W. A. Davis, ``Amplitude modulator utilizing a
%high-Q class-E DC-DC converter'', \emph {2003 IEEE MTT-S Int. Microwave
%Symp. Dig.}, vol. 3, pp. 1721-1724, June 2003.

\bibitem {krauss}
H. L. Krauss, C. W. Bostian, and F. H. Raab, \emph{Solid State Radio Engineering}, New York: J. Wiley \& Sons, 1980.

\bibitem{1} Hsu, M.L., Chiu, K.Y.: A comparison between I-Ching’s early management decision-making model and western management decision-making models. Chin. Manag. Stud. 2(1), 52–75 (2008)

\bibitem{2} Ban, G., Ban, Z.: Book of Han, Treatise on Literature. Shanghai Classics Publishing House, Shanghai (2009)

\bibitem{3} Wang, B., Han, K., Kong, Y.: ZhouYi Zhengyi. Dahua shuju, Taipei (1989)

\bibitem{4} Lu, Y., Qian, Y., Wang, D., Lu, Y.: Implications of I Ching on innovation management. Chin. Manag. Stud.5(4), 394–402 (2011)

\bibitem{5} Ho, L.S., Chen, S.Y. and Yen, T.M.: Using Yi-Jing science name classification model in 2014 Taiwan City
Mayors and Members Election. In: Conference of the first I Ching Chinese Culture and Cultural Creative
management, Guangxi, January, 66–73 (2016)

\bibitem{6} Yu, D.H. (1987). About happiness: A Chinese perspective. Taipei: Teacher Chang Press.

\bibitem{7} Lu, L.Z. (1990). Heaven, human, and society: A study of the traditional Chinese cognitive model
of the universe. Taipei: Institute of Nationality, Academia Sinica.

\bibitem{8} Qu, H.Y. (1999). Popularity of folk practices of fortune telling and social change. Journal of
Taiwan Sociology, 22, 1–45.

\bibitem{9} ETDA Research (2017), “Behavior on Using Internet 2017,” accessed October 4,2017
at http://thumbsup.in.th/2017/09/etda-internet-profile-thailand-2017/

\bibitem{10} Kasikorn Thai Research (2009), “Bangkok and Fortune Telling Service,” accessed
October 2, 2017 at https://www.kasikornresearch.com/th/keconanalysis/pages/
.aspx?docid=6624

\bibitem{11} Sinthuwongsri, A. (2017), “The fortune Telling Behavior of City Life in Bangkok,”
Social Research Journal, accessed October 2, 2017 at
http://www.socialresearchjournal.com/documents/archive/journal-60-01-
08.pdf

\bibitem{12} Piller, F. T., \& Walcher, D. (2006), “Toolkits for idea competitions: A novel method
to integrate users in new,” R\&D Management, 36(3), 210-213.

\bibitem{13} Heinonen, K. (2011), “Consumer activity in social media: Managerial approaches to
consumers' social media behavior,” Journal of Consumer Behaviour, 10, 356–364.

\bibitem{14} Turcotte, J., York, C., Irving, J., Scholl, R. M., \& Pingree, R. J. (2015). “News
recommendations from social media opinion leaders: Effects on media trust
and information seeking,” Journal of Computer-Mediated Communications,
520–535.

\bibitem{15} Stigler, G. J. (1961), “The Economics of Information,” Journal of Political Economy
(69:3), 213-225.

\bibitem{16} Johnson, E., and Payne, J. 1985. “Effort and Accuracy in Choice,” Management
Science, 395-415.

\bibitem{17} Nelson, P. 1970. “Information and Consumer Behavior,” Journal of Political
Economy (78:20), 311-329.

\bibitem{18} Huang, P., Lurie, N. H., and Mitra, S. 2009. “Searching for Experience on the Web:
An Empirical Examination of Consumer Behavior for Search and Experience
Goods,” Journal of Marketing (73:2), 55-69.

\bibitem{20} Kern, Martin (2010). "Early Chinese literature, Beginnings through Western Han". In Owen, Stephen (ed.). The Cambridge History of Chinese Literature, Volume 1: To 1375. Cambridge, England: Cambridge University Press. pp. 1–115. ISBN 978-0-521-11677-0.

\bibitem{21} Smith, Richard J. (2012). The I Ching: A Biography. Princeton: Princeton University Press. ISBN 978-0-691-14509-9.

\bibitem{22} Nylan, Michael (2001). The Five "Confucian" Classics. New Haven: Yale University Press. ISBN 0-300-13033-3.

\bibitem{23} Smith, Richard J. (2008). Fathoming the Cosmos and Ordering the World: the Yijing (I Ching, or Classic of Changes) and its Evolution in China. Charlottesville: University of Virginia Press. ISBN 978-0-8139-2705-3.


%\bibitem{IEEEhowto:kopka}
%H.~Kopka and P.~W. Daly, \emph{A Guide to {\LaTeX}}, 3rd~ed.\hskip 1em plus
% 0.5em minus 0.4em\relax Harlow, England: Addison-Wesley, 1999.

%\bibitem{lamport} L. Lamport, \emph{ {\LaTeX} A Document Preparation
%  System}, Reading, Mass: Addison-Wesley, 1994.

%\bibitem{knuth} D. E. Knuth, \emph {The \TeX book}, Reading, Mass.:
%  Addison-Wesley, 1996.

\end{thebibliography}
%\smallskip
%Note: For the Summary paper submission only, references to the authors own work should be cited as if done by others to enable a double-blind review. {\bfseries Citations must be complete and not redacted, allowing the reviewers to confirm that prior art has been properly identified and acknowledged.}
% that's all folks
\end{document}